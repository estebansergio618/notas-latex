\documentclass[10pt]{book}

\usepackage[utf8]{inputenc}
\usepackage[spanish]{babel}
\usepackage{geometry}
\usepackage{hyperref}
% =================================================================
\usepackage[x11names]{xcolor}
% https://ctan.dcc.uchile.cl/macros/latex/contrib/xcolor/xcolor.pdf
% Página39

% ================CIRCULO TRIGONOMETRICO===========================
\usepackage{tikz, tkz-euclide}
% TikZ: https://www.ctan.org/pkg/pgf
% tkz-euclide: https://www.ctan.org/pkg/tkz-euclide

% =================================================================
\usepackage{calculator}
% % calculator: https://www.ctan.org/pkg/calculator

% =====================DIAGRAMAS DE VENN===========================
\usepackage{venndiagram}
% https://ctan.dcc.uchile.cl/macros/latex/contrib/venndiagram/venndiagram.pdf
% =================================================================

% \geometry{a4paper, top=1.5cm, bottom=1cm}
\title{TikZ \\ \normalsize Notas en \LaTeX}
\author{Sergio Esteban Muñoz \\ \href{mailto: semunoz@dc.uba.ar}{\nolinkurl{semunoz@dc.uba.ar}}} 
%\and Autor 2}
\date{\today}

\begin{document}
\maketitle

% =================================================================
\chapter{Geometía con \LaTeX}
Empiezo con Diagramas de Venn que utlice para escribir el solucionario de la \textbf{Práctica N°4} cuando curse \textbf{Algebra I}, luego con un programa para ver graficamente las raices de la unidad y finalmente me metere de lleno con el uso de los paquetes \texttt{tikz y tkz-euclide} para la grafica de figuras geométricas, el código estara en mi github para que puedas ver el código.

\section{Diagramas de Venn}
% ACapB=A intersección B
% NotA = complemento de A
% shade = sombra
% label=etiqueta
% scale sucede en las letras
    \begin{venndiagram2sets}[shade=DeepPink1, radius=1.5cm, labelOnlyA=$x_1$]
        \fillANotB
    \end{venndiagram2sets}
\hspace{1cm}
    \begin{venndiagram3sets}[labelA=$T_1$, labelOnlyA=1, labelOnlyB=189, labelABC=1, labelNotABC=21, shade=LightSkyBlue2, radius=1cm, tikzoptions={scale=1.5, font=\large, draw=red}]
        \fillACapBNotC
    \end{venndiagram3sets}
    
% =================================================================

\section{Circunferencia Trigonométrica(C.T)}
	\begin{itemize}
	    \item Se realizó un programa para la CT en \LaTeX. (\texttt{GitHub:@estebansergio618}) 
	    \item En $\backslash$\texttt{def} $\backslash$\texttt{partes\{12\}} podemos cambiar en cuanto queremos partir la CT. (Máximo 12)
	\end{itemize}
	\\
	\begin{tikzpicture}[scale=3]
	
	\tkzInit[xmin=-1.2, xmax=1, ymin=-1.2, ymax=1]
	\tkzDrawXY[thick,noticks]
	
	\tkzDefPoints{0/0/O}
	
	\tkzDrawCircle[thin,R](O,1cm)
	
	\def\partes{12} 
	\def\tipo{2}


	\foreach \i in {1,2,...,\partes} {
	\tkzDefPoint({cos(\i*2*pi/\partes)},{sin(\i*2*pi/\partes)}){P\i}
	\tkzDrawPoint(P\i)
	\tkzDrawSegment[ultra thin, dashed](O,P\i)
		
		
	\MULTIPLY{2}{\i}{\pnumI}
	\FRACTIONSIMPLIFY{\pnumI}{\partes}{\pnumII}{\pdenII}
		
	\ifnum\pdenII=1
		\ifnum\pnumII=1
			\tkzLabelSegment[above=-6pt, pos=0.8,fill=white](O,P\i){\(\pi\)}
		\else
			\tkzLabelSegment[above=-13pt, pos=0.8,fill=white](O,P\i){\(\pnumII \pi\)}
			\tkzLabelSegment[above=-1pt, pos=0.8,fill=white](O,P\i){\(0\)}
		\fi
	\else
		\ifnum\pnumII=1
			\tkzLabelSegment[above=-10pt, pos=0.8,fill=white](O,P\i){\(\frac{\pi}{\pdenII}\)}
		\else
			\tkzLabelSegment[above=-10pt, pos=0.8,fill=white](O,P\i){\(\frac{\pnumII\pi}{\pdenII}\)}
		\fi
	\fi
	}
	\else
		\ifnum\tipo=2
		
	\end{tikzpicture}
% =================================================================

\section{Triangulos (tkz-euclide)}

    \begin{tikzpicture}[scale=2]
    
        % DEFINIR PUNTOS:
        \tkzDefPoint(0,0){A}
        \tkzDefPoint(4,0){B}
        % coordenadas polares para definir:
        \tkzDefPoint(130:3){C}
        
        % DIBUJAR SEGMENTOS
        \tkzDrawPolygon(A,B,C)
        \tkzDefLine[orthogonal=through C](A,B)\tkzGetPoint{c}
        % TKZ INTER SECCION LINEA A LINEA:
        \tkzInterLL(A,B)(C,c)\tkzGetPoint{D}
        
        
        \tkzMarkAngle[size=.5, arrows= ->,>=stealth, color=red](B,A,C)
        \tkzMarkAngle[size=.4, arrows= ->,>=stealth, color=blue](C,A,D)
        \tkzMarkRightAngle[size=.3, color=blue, fill=blue!20, draw opacity=0](A,D,C)
        
        \tkzDrawSegment[dashed](C,D)
        \tkzDrawLine[add=0cm and 1cm, dashed](A,D)
        
        % DIBUJAR PUNTOS
        \tkzDrawPoints(A,B,C,D)
        \tkzLabelPoints[below](A,D)
        \tkzLabelPoints[right](B)
        \tkzLabelPoints[above](C)
        
        \begin{scope}{font=\tiny}
        
        \tkzLabelSegment[below](A,B){$c$}
        \tkzLabelSegment(A,C){$b$}
        \tkzLabelSegment[above](B,C){$a$}
        \tkzLabelSegment[left](C,D){$h$}
        \tkzLabelAngle[pos=.7](B,A,C){$\widehat{A}$}
        \tkzLabelAngle[pos=.5, left](C,A,D){$\pi-\widehat{A}$}
        
        \end{scope}
        
        
    \end{tikzpicture}
    
    \subsection{Diseño de Polígonos Regulares}
    
    \begin{tikzpicture}
    % sides=lados (lo puedo cambiar)
        \tkzDefPoints(0/0/A, 1/0/B)
        
        \tkzDefRegPolygon[side,sides=5](A,B)\tkzGetPoint{O}
        \tkzLabelRegPolygon(O){A,...,E}
        \tkzDrawPolygon(P1,P2,P3,P4,P5)
        \tkzDrawPoints(A,B)
    \end{tikzpicture}
    















\end{document}
